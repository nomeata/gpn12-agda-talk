\documentclass[12pt,parskip=half,headings=normal,abstract]{scrartcl}

\usepackage[T3,T1]{fontenc}
\usepackage[utf8]{inputenc}
\usepackage[german]{babel}
\usepackage{amsmath}
\usepackage{amsfonts}
\usepackage[standard]{ntheorem}
\usepackage{amscd}
\usepackage{mathtools}
\usepackage{stmaryrd}
\usepackage{faktor}
\usepackage{enumerate}
\usepackage{wrapfig}
\usepackage[safe]{tipa} % for \textlambda
\usepackage{hevea}

\usepackage[pdfauthor={Joachim Breitner},pdftitle={Agda: Mit starken Typen abhängen},pdfsubject={Eine Demo der abhängig getypten Programmiersprache}]{hyperref}
\usepackage{listings}
\usepackage{multicol}
\usepackage{ragged2e}
%\usepackage[para]{footmisc}
\usepackage{fourier}
\usepackage{microtype}
\usepackage{comment}

\usepackage{tikz}
\usetikzlibrary{shadows}

\newstyle{.keystroke}{display:inline;border:1px solid black;padding:1px;background:light gray}
\newcommand*\keystroke[1]{%
\begin{latexonly}%
  \tikz[baseline=(key.base)]
    \node[%
      draw,
      fill=white,
      drop shadow={shadow xshift=0.25ex,shadow yshift=-0.25ex,fill=black,opacity=0.75},
      rectangle,
      rounded corners=2pt,
      inner sep=1pt,
      line width=0.5pt,
      font=\scriptsize\sffamily
    ](key) {#1\strut}
  ;%
\end{latexonly}%
\ifhevea%
\@span{class="keystroke"}%
#1%
\@clearstyle %
\fi%
}

\newcommand{\agdastroke}[1]{\keystroke{Strg}\keystroke C\keystroke #1}



\urlstyle{sf}

\author{Joachim Breitner\footnote{\href{mailto:mail@joachim-breitner.de}{mail@joachim-breitner.de}, \url{http://www.joachim-breitner.de/}. Diese Arbeit wurde durch ein Promotionsstipendium der \href{http://telekom-stiftung.de/}{Deutschen Telekom Stiftung} gefördert.}}
\title{Agda: Mit starken Typen abhängen}
\subject{Eine Demo der abhängig getypten Programmiersprache}
\publishers{Gulaschprogrammiernacht 12\footnote{\url{http://entropia.de/GPN12}}, Karlsruhe}
\date{7. Juni 2012}

\lstset{
	language=Haskell
	,literate=
		{ö}{{\"o}}1
	        {ä}{{\"a}}1
                {ü}{{\"u}}1
                {ℕ}{{$\mathbb N$}}1
                {≤}{{$\le$}}1
                {-}{{\text-}}1
                {≡}{{$\equiv$}}1
                {≢}{{$\not\equiv$}}1
                {∎}{{$\blacksquare$}}1
                {□}{{$\square$}}1
                {⟨}{{$\langle$}}1
                {⟩}{{$\rangle$}}1
                {→}{{$\rightarrow$}}1
                {⇒}{{$\Rightarrow$}}1
                {∸}{{$\stackrel{\bullet}{-}$}}1
                {∷}{{$::$}}1
                {∣}{{$\mid$}}1
                {¬}{{$\neg$}}1
                {∀}{{$\forall$}}1
                {⊥}{{$\bot$}}1
                {λ}{{\textlambda}}1
                {′}{{'}}1
        ,columns=flexible
        ,basewidth={.365em}
        ,keepspaces=True
        ,texcl=true
        ,basicstyle=\small\sffamily
        ,stringstyle=\itshape
        ,showstringspaces=false
        ,keywords={module,where,open,import,using,renaming,to,data,let,in,with}
	,rangeprefix=\{-L 
	,rangesuffix=-\}
	,includerangemarker=false
	,belowskip=0pt
}
\lstnewenvironment{code}{}{}
\newcommand{\li}{\lstinline}


\begin{document}

\maketitle

Andrew Hunt und David Thomas empfehlen
\begin{quote}
„Lerne mindestens eine neue Programmiersprache pro Jahr.“
\end{quote}
Ganz in diesem Sinne widmen wir uns heute der Programmiersprache Agda. Agda ist zum einen eine funktionale Programmiersprache wie Haskell oder ML. Das alleine wäre jetzt nicht so spannend, aber Agda hat ein sehr mächtiges Typsystem mit sogenannten abhängigen Typen (\emph{dependent types}), die dem Programmierer erlauben sehr viele Aussagen über sein Programm als Typen anzugeben und vom Compiler prüfen zu lassen. Das geht sogar so weit dass Agda auch ein Beweisprüfer wie Coq oder Isabelle ist, mit dem sich (konstruktive) Mathematik betreiben lässt.

Dass das geht, und vor allem dass das ziemlich gut geht, liegt an folgender „verhält-sich-zu-wie-Relation“: Programme zu Typen wie Beweise zu Aussagen. Eine Aussage wie
\begin{quote}
Wenn aus $\alpha$ $\beta$ folgt, dann folgt,
wenn $\beta$ nicht gilt,  auch
dass $\alpha$ nicht gilt.
\end{quote}
hat einen Beweis, etwa
\begin{quote}
Es gelte $\beta$ nicht. Beweis per Widerspruch. Angenommen, es
gelte $\alpha$. Dann folgt aus der Voraussetzung, dass $\beta$ gilt.
Das ist ein Widerspruch dazu, dass $\beta$ nicht
gilt, womit die Aussage gezeigt ist.
$\blacksquare$
\end{quote}
Der Aussage entspricht der Typ
\begin{quote}
\li!(a → b) → ¬ b → ¬ a!
\end{quote}
und dem Beweis entspricht das Programm
\begin{quote}
\li!λ f x a. x (f a)!.
\end{quote}
Ein Typ, zu dem es ein Programm gibt, entspricht damit einer bewiesenen Aussage und Beweisen ist plötzlich nichts anderes mehr als Programmieren. Wir sind als auf der Gulaschbeweisnacht!  Das Ganze ist übrigens bekannt als der Curry-Howard-Isomorphismus.

In diesem Vortrag werden wir beide Aspekte von Agda sehen, als Beispiel dient hier ein kleines Spiel:

\begin{quote}
Auf dem Tisch liegt eine Anzahl Murmeln. Abwechselnd nimmt jeder Spieler ein paar Murmeln; mindestens eine und höchstens sechs. Wer die letzte Murmel nimmt, verliert.
\end{quote}

\input{Pebbels1.lagda}
\input{Pebbels3.lagda}
\input{Pebbels5.lagda}

\appendix
\input{Pebbels6.lagda}

\end{document}

